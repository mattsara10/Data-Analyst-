% Options for packages loaded elsewhere
\PassOptionsToPackage{unicode}{hyperref}
\PassOptionsToPackage{hyphens}{url}
%
\documentclass[
]{article}
\usepackage{amsmath,amssymb}
\usepackage{iftex}
\ifPDFTeX
  \usepackage[T1]{fontenc}
  \usepackage[utf8]{inputenc}
  \usepackage{textcomp} % provide euro and other symbols
\else % if luatex or xetex
  \usepackage{unicode-math} % this also loads fontspec
  \defaultfontfeatures{Scale=MatchLowercase}
  \defaultfontfeatures[\rmfamily]{Ligatures=TeX,Scale=1}
\fi
\usepackage{lmodern}
\ifPDFTeX\else
  % xetex/luatex font selection
\fi
% Use upquote if available, for straight quotes in verbatim environments
\IfFileExists{upquote.sty}{\usepackage{upquote}}{}
\IfFileExists{microtype.sty}{% use microtype if available
  \usepackage[]{microtype}
  \UseMicrotypeSet[protrusion]{basicmath} % disable protrusion for tt fonts
}{}
\makeatletter
\@ifundefined{KOMAClassName}{% if non-KOMA class
  \IfFileExists{parskip.sty}{%
    \usepackage{parskip}
  }{% else
    \setlength{\parindent}{0pt}
    \setlength{\parskip}{6pt plus 2pt minus 1pt}}
}{% if KOMA class
  \KOMAoptions{parskip=half}}
\makeatother
\usepackage{xcolor}
\usepackage[margin=1in]{geometry}
\usepackage{color}
\usepackage{fancyvrb}
\newcommand{\VerbBar}{|}
\newcommand{\VERB}{\Verb[commandchars=\\\{\}]}
\DefineVerbatimEnvironment{Highlighting}{Verbatim}{commandchars=\\\{\}}
% Add ',fontsize=\small' for more characters per line
\usepackage{framed}
\definecolor{shadecolor}{RGB}{248,248,248}
\newenvironment{Shaded}{\begin{snugshade}}{\end{snugshade}}
\newcommand{\AlertTok}[1]{\textcolor[rgb]{0.94,0.16,0.16}{#1}}
\newcommand{\AnnotationTok}[1]{\textcolor[rgb]{0.56,0.35,0.01}{\textbf{\textit{#1}}}}
\newcommand{\AttributeTok}[1]{\textcolor[rgb]{0.13,0.29,0.53}{#1}}
\newcommand{\BaseNTok}[1]{\textcolor[rgb]{0.00,0.00,0.81}{#1}}
\newcommand{\BuiltInTok}[1]{#1}
\newcommand{\CharTok}[1]{\textcolor[rgb]{0.31,0.60,0.02}{#1}}
\newcommand{\CommentTok}[1]{\textcolor[rgb]{0.56,0.35,0.01}{\textit{#1}}}
\newcommand{\CommentVarTok}[1]{\textcolor[rgb]{0.56,0.35,0.01}{\textbf{\textit{#1}}}}
\newcommand{\ConstantTok}[1]{\textcolor[rgb]{0.56,0.35,0.01}{#1}}
\newcommand{\ControlFlowTok}[1]{\textcolor[rgb]{0.13,0.29,0.53}{\textbf{#1}}}
\newcommand{\DataTypeTok}[1]{\textcolor[rgb]{0.13,0.29,0.53}{#1}}
\newcommand{\DecValTok}[1]{\textcolor[rgb]{0.00,0.00,0.81}{#1}}
\newcommand{\DocumentationTok}[1]{\textcolor[rgb]{0.56,0.35,0.01}{\textbf{\textit{#1}}}}
\newcommand{\ErrorTok}[1]{\textcolor[rgb]{0.64,0.00,0.00}{\textbf{#1}}}
\newcommand{\ExtensionTok}[1]{#1}
\newcommand{\FloatTok}[1]{\textcolor[rgb]{0.00,0.00,0.81}{#1}}
\newcommand{\FunctionTok}[1]{\textcolor[rgb]{0.13,0.29,0.53}{\textbf{#1}}}
\newcommand{\ImportTok}[1]{#1}
\newcommand{\InformationTok}[1]{\textcolor[rgb]{0.56,0.35,0.01}{\textbf{\textit{#1}}}}
\newcommand{\KeywordTok}[1]{\textcolor[rgb]{0.13,0.29,0.53}{\textbf{#1}}}
\newcommand{\NormalTok}[1]{#1}
\newcommand{\OperatorTok}[1]{\textcolor[rgb]{0.81,0.36,0.00}{\textbf{#1}}}
\newcommand{\OtherTok}[1]{\textcolor[rgb]{0.56,0.35,0.01}{#1}}
\newcommand{\PreprocessorTok}[1]{\textcolor[rgb]{0.56,0.35,0.01}{\textit{#1}}}
\newcommand{\RegionMarkerTok}[1]{#1}
\newcommand{\SpecialCharTok}[1]{\textcolor[rgb]{0.81,0.36,0.00}{\textbf{#1}}}
\newcommand{\SpecialStringTok}[1]{\textcolor[rgb]{0.31,0.60,0.02}{#1}}
\newcommand{\StringTok}[1]{\textcolor[rgb]{0.31,0.60,0.02}{#1}}
\newcommand{\VariableTok}[1]{\textcolor[rgb]{0.00,0.00,0.00}{#1}}
\newcommand{\VerbatimStringTok}[1]{\textcolor[rgb]{0.31,0.60,0.02}{#1}}
\newcommand{\WarningTok}[1]{\textcolor[rgb]{0.56,0.35,0.01}{\textbf{\textit{#1}}}}
\usepackage{longtable,booktabs,array}
\usepackage{calc} % for calculating minipage widths
% Correct order of tables after \paragraph or \subparagraph
\usepackage{etoolbox}
\makeatletter
\patchcmd\longtable{\par}{\if@noskipsec\mbox{}\fi\par}{}{}
\makeatother
% Allow footnotes in longtable head/foot
\IfFileExists{footnotehyper.sty}{\usepackage{footnotehyper}}{\usepackage{footnote}}
\makesavenoteenv{longtable}
\usepackage{graphicx}
\makeatletter
\def\maxwidth{\ifdim\Gin@nat@width>\linewidth\linewidth\else\Gin@nat@width\fi}
\def\maxheight{\ifdim\Gin@nat@height>\textheight\textheight\else\Gin@nat@height\fi}
\makeatother
% Scale images if necessary, so that they will not overflow the page
% margins by default, and it is still possible to overwrite the defaults
% using explicit options in \includegraphics[width, height, ...]{}
\setkeys{Gin}{width=\maxwidth,height=\maxheight,keepaspectratio}
% Set default figure placement to htbp
\makeatletter
\def\fps@figure{htbp}
\makeatother
\setlength{\emergencystretch}{3em} % prevent overfull lines
\providecommand{\tightlist}{%
  \setlength{\itemsep}{0pt}\setlength{\parskip}{0pt}}
\setcounter{secnumdepth}{-\maxdimen} % remove section numbering
\ifLuaTeX
  \usepackage{selnolig}  % disable illegal ligatures
\fi
\usepackage{bookmark}
\IfFileExists{xurl.sty}{\usepackage{xurl}}{} % add URL line breaks if available
\urlstyle{same}
\hypersetup{
  pdftitle={What Predicts Good 3PT Shooting?},
  pdfauthor={Matt Saraceno},
  hidelinks,
  pdfcreator={LaTeX via pandoc}}

\title{What Predicts Good 3PT Shooting?}
\author{Matt Saraceno}
\date{2024-07-15}

\begin{document}
\maketitle

\subsubsection{\texorpdfstring{\textbf{Introduction}}{Introduction}}\label{introduction}

Three point shooting(3PT) is a critical part of modern basketball. The
best teams in the National Basketball Associate(NBA) are all in an arms
race to load up on the best three point shooters in the world. With so
many players being evaluated on their ability to shoot the three ball,
what data can be used to help predict a players ability to make three
pointers?

The goal of this report is to predict a players ability to hit three
pointers by looking at three different factors: free-throw(FT)
percentage, mid-range percentage, and the position they play.

\subsubsection{\texorpdfstring{\textbf{Question \#1: Does the ability to
make free-throws influence three-point
shooting?}}{Question \#1: Does the ability to make free-throws influence three-point shooting?}}\label{question-1-does-the-ability-to-make-free-throws-influence-three-point-shooting}

We have all heard this before in one way or another, ``A player's
ability to make free throws is an indicator that a player is, or can be,
a great shooter''. This line of thinking has been repeated so many times
across sports media that it has become basketball common knowledge, but
is it true?

I have decided to take a look at last years shooting totals from all 572
NBA players, and see if there truly is an correlation between the two
statistics.

The code below gives a quick preview of what our data looks like.

\begin{Shaded}
\begin{Highlighting}[]
\FunctionTok{library}\NormalTok{(tidyverse)}
\end{Highlighting}
\end{Shaded}

\begin{verbatim}
## -- Attaching core tidyverse packages ------------------------ tidyverse 2.0.0 --
## v dplyr     1.1.4     v readr     2.1.5
## v forcats   1.0.0     v stringr   1.5.1
## v ggplot2   3.5.1     v tibble    3.2.1
## v lubridate 1.9.3     v tidyr     1.3.1
## v purrr     1.0.2     
## -- Conflicts ------------------------------------------ tidyverse_conflicts() --
## x dplyr::filter() masks stats::filter()
## x dplyr::lag()    masks stats::lag()
## i Use the conflicted package (<http://conflicted.r-lib.org/>) to force all conflicts to become errors
\end{verbatim}

\begin{Shaded}
\begin{Highlighting}[]
\FunctionTok{library}\NormalTok{(readxl)}
\NormalTok{mydata1 }\OtherTok{\textless{}{-}} \FunctionTok{read\_excel}\NormalTok{(}\StringTok{"D:/Data Analyst/3pt shooting/3pt\_shooting.xlsx"}\NormalTok{)}
\FunctionTok{head}\NormalTok{(mydata1)}
\end{Highlighting}
\end{Shaded}

\begin{verbatim}
## # A tibble: 6 x 30
##      Rk Player    Pos     Age Tm        G    GS    MP    FG   FGA FG_per three_P
##   <dbl> <chr>     <chr> <dbl> <chr> <dbl> <dbl> <dbl> <dbl> <dbl>  <dbl>   <dbl>
## 1     1 Precious~ PF       24 TOT      74    18  1624   235   469  0.501      26
## 2     2 Bam Adeb~ C        26 MIA      71    71  2416   530  1017  0.521      15
## 3     3 Ochai Ag~ SG       23 TOT      78    28  1641   178   433  0.411      62
## 4     4 Santi Al~ PF       23 MEM      61    35  1618   247   568  0.435     106
## 5     5 Nickeil ~ SG       25 MIN      82    20  1921   236   538  0.439     131
## 6     6 Grayson ~ SG       28 PHO      75    74  2513   340   682  0.499     205
## # i 18 more variables: three_PA <dbl>, threeP_percent <dbl>, two_P <dbl>,
## #   two_PA <dbl>, twoP_per <dbl>, eFG_per <dbl>, FT <dbl>, FTA <dbl>,
## #   FT_percent <dbl>, ORB <dbl>, DRB <dbl>, TRB <dbl>, AST <dbl>, STL <dbl>,
## #   BLK <dbl>, TOV <dbl>, PF <dbl>, PTS <dbl>
\end{verbatim}

Before we can start to do any analysis, we must first filter out some
data. Not every NBA player this season shot enough FTs or three-pointers
to give us any meaningful data. This report will look at only players
that have attempted at least 150 FTs and attempted at least 85 three
pointers.

Below is a graph of charting the two statistics:

\begin{Shaded}
\begin{Highlighting}[]
\NormalTok{filtered\_mydata1 }\OtherTok{\textless{}{-}}\NormalTok{ mydata1[mydata1}\SpecialCharTok{$}\NormalTok{FTA }\SpecialCharTok{\textgreater{}=} \DecValTok{150} \SpecialCharTok{\&}\NormalTok{ mydata1}\SpecialCharTok{$}\NormalTok{three\_PA }\SpecialCharTok{\textgreater{}=}\DecValTok{85}\NormalTok{, ]}
\FunctionTok{ggplot}\NormalTok{(filtered\_mydata1, }\FunctionTok{aes}\NormalTok{(}\AttributeTok{x =}\NormalTok{ FT\_percent, }\AttributeTok{y =}\NormalTok{ threeP\_percent))}\SpecialCharTok{+}
  \FunctionTok{geom\_point}\NormalTok{()}\SpecialCharTok{+}
  \FunctionTok{ggtitle}\NormalTok{(}\StringTok{"3PT\% based on FT\%"}\NormalTok{)}\SpecialCharTok{+}
  \FunctionTok{xlab}\NormalTok{(}\StringTok{"FT\%"}\NormalTok{)}\SpecialCharTok{+}
  \FunctionTok{ylab}\NormalTok{(}\StringTok{"3PT\%"}\NormalTok{)}\SpecialCharTok{+}
  \FunctionTok{geom\_smooth}\NormalTok{(}\AttributeTok{method =} \StringTok{\textquotesingle{}lm\textquotesingle{}}\NormalTok{)}
\end{Highlighting}
\end{Shaded}

\begin{verbatim}
## `geom_smooth()` using formula = 'y ~ x'
\end{verbatim}

\includegraphics{Report_files/figure-latex/unnamed-chunk-2-1.pdf}

Out of the 572 initial players that were observed, only 106 made the cut
after our filtering. The median FT percentage from this group of players
is 81.62\% and their 3PT shooting percentage is 36.03\%. Those are some
pretty solid numbers from some of the best players in the world.

The initial thoughts from examining this graph, is that there seems to
be a positive correlation between the two stats. A trend line is added
to the graph to provide the audience the direction of the data. The data
is not heavily correlated due to the random nature that the data
presents. That is reinforced by our correlation coefficient.

\begin{Shaded}
\begin{Highlighting}[]
\FunctionTok{cor}\NormalTok{(filtered\_mydata1}\SpecialCharTok{$}\NormalTok{FT\_percent, filtered\_mydata1}\SpecialCharTok{$}\NormalTok{threeP\_percent)}
\end{Highlighting}
\end{Shaded}

\begin{verbatim}
## [1] 0.5059518
\end{verbatim}

Correlation coefficient provides a measurement to determine if there is
any significant relationship between two variables. The closer the
correlation coefficient is to 1, the stronger the relationship is. The
correlation coefficient for this data is .505918. A correlation
coefficient of less than .7 is not strong enough to infer that high FT
percentage is linked to good three point shooting.

\subsubsection{\texorpdfstring{\textbf{Question \#2: Does a player's
ability to make mid-range jump shots influence a player ability to make
three-pointers?}}{Question \#2: Does a player's ability to make mid-range jump shots influence a player ability to make three-pointers?}}\label{question-2-does-a-players-ability-to-make-mid-range-jump-shots-influence-a-player-ability-to-make-three-pointers}

The next variable of interest is the mid range jump shots. For this
study, shooting data by distance was used to help determine mid range
shoots. Shoots taken between 10-16 feet was labeled as a mid-range. Just
like with the free throw study above, the data used to answer this
question was from shooting data of 572 NBA players from the 2023-24
season. Then the data was filtered to look at players who took at least
10\% of the shots from 10-16 feet and played at least 25 games. Some
outliers were removed to clean up the data as well.

\begin{Shaded}
\begin{Highlighting}[]
\NormalTok{mydata2 }\OtherTok{\textless{}{-}} \FunctionTok{read\_excel}\NormalTok{(}\StringTok{"D:/Data Analyst/3pt shooting/shooting\_advanced.xlsx"}\NormalTok{)}
\NormalTok{filtered\_mydata2 }\OtherTok{\textless{}{-}}\NormalTok{ mydata2[mydata2}\SpecialCharTok{$}\NormalTok{ten\_sixteenAperbyDistance }\SpecialCharTok{\textgreater{}=}\NormalTok{ .}\DecValTok{125} \SpecialCharTok{\&}\NormalTok{ mydata2}\SpecialCharTok{$}\NormalTok{Games }\SpecialCharTok{\textgreater{}=} \DecValTok{25}\NormalTok{, ]}
\NormalTok{filtered\_mydata3 }\OtherTok{\textless{}{-}}\NormalTok{ filtered\_mydata2 }\SpecialCharTok{\%\textgreater{}\%}  \FunctionTok{filter}\NormalTok{(}\SpecialCharTok{!}\FunctionTok{row\_number}\NormalTok{() }\SpecialCharTok{\%in\%} \FunctionTok{c}\NormalTok{(}\DecValTok{53}\NormalTok{, }\DecValTok{68}\NormalTok{))}
\NormalTok{filtered\_mydata4 }\OtherTok{\textless{}{-}} \FunctionTok{subset}\NormalTok{(filtered\_mydata3, }\SpecialCharTok{!}\FunctionTok{is.na}\NormalTok{(filtered\_mydata3}\SpecialCharTok{$}\NormalTok{threep\_fgperbyDistance))}
\FunctionTok{head}\NormalTok{(filtered\_mydata4)}
\end{Highlighting}
\end{Shaded}

\begin{verbatim}
## # A tibble: 6 x 29
##      Rk Player     Pos     Age Tm    Games    MP `FG%` Dist. twoPA_perbyDistance
##   <dbl> <chr>      <chr> <dbl> <chr> <dbl> <dbl> <dbl> <dbl>               <dbl>
## 1     2 Bam Adeba~ C        26 MIA      71  2416 0.521   8.5               0.959
## 2    10 Kyle Ande~ PF       30 MIN      79  1782 0.46    9                 0.889
## 3    13 Cole Anth~ PG       23 ORL      81  1817 0.435  14.9               0.64 
## 4    17 Deandre A~ C        25 POR      55  1784 0.57    8.9               0.987
## 5    25 Paolo Ban~ PF       21 ORL      80  2799 0.455  12.6               0.751
## 6    28 Dominick ~ PF       20 SAS      33   420 0.496   7.4               0.92 
## # i 19 more variables: zero_3AperbyDistance <dbl>,
## #   three_tenAperbyDistance <dbl>, ten_sixteenAperbyDistance <dbl>,
## #   sixteen_threepAperbyDistance <dbl>, threepAperbyDistance <dbl>,
## #   twop_fgperbyDistance <dbl>, zero_threefgperbyDistnace <dbl>,
## #   three_tenfgperbyDistance <dbl>, ten_sixteenfgperbyDistance <dbl>,
## #   sixteen_threepfgperbyDistance <dbl>, threep_fgperbyDistance <dbl>,
## #   twop_perass <dbl>, threep_perass <dbl>, dunks_perFGA <dbl>, ...
\end{verbatim}

After the filtering is applied, we are left with 74 players that will be
elevated to determine if the ability to make mid range shots has any
influence on a players ability to hit three pointers. We will replicate
the same exercise that we did in question \#1. First, we will plot the
data and then we will look at the correlation coefficient to determine
if their is any relationship between the two variables.

\begin{Shaded}
\begin{Highlighting}[]
\FunctionTok{ggplot}\NormalTok{(filtered\_mydata4, }\FunctionTok{aes}\NormalTok{(}\AttributeTok{x =}\NormalTok{ ten\_sixteenfgperbyDistance, }\AttributeTok{y =}\NormalTok{ threep\_fgperbyDistance))}\SpecialCharTok{+}
  \FunctionTok{geom\_point}\NormalTok{(}\FunctionTok{aes}\NormalTok{(}\AttributeTok{color =}\NormalTok{ Pos))}\SpecialCharTok{+}
  \FunctionTok{ggtitle}\NormalTok{(}\StringTok{"3PT\% based on \%of mid range attempts"}\NormalTok{)}\SpecialCharTok{+}
  \FunctionTok{xlab}\NormalTok{(}\StringTok{"\% 10{-}16ft"}\NormalTok{)}\SpecialCharTok{+}
  \FunctionTok{ylab}\NormalTok{(}\StringTok{"Three Point \%"}\NormalTok{)}\SpecialCharTok{+}
  \FunctionTok{geom\_hline}\NormalTok{(}\AttributeTok{yintercept =}\NormalTok{ .}\DecValTok{35}\NormalTok{)}\SpecialCharTok{+}
  \FunctionTok{geom\_vline}\NormalTok{(}\AttributeTok{xintercept =}\NormalTok{ .}\DecValTok{45}\NormalTok{)}\SpecialCharTok{+}
  \FunctionTok{annotate}\NormalTok{(}\StringTok{"text"}\NormalTok{, }\AttributeTok{x =}\NormalTok{.}\DecValTok{55}\NormalTok{, }\AttributeTok{y =}\NormalTok{.}\DecValTok{475}\NormalTok{, }\AttributeTok{label =} \StringTok{"Great Shooters"}\NormalTok{, }\AttributeTok{col =} \StringTok{"red"}\NormalTok{, }\AttributeTok{size =} \DecValTok{4}\NormalTok{)}\SpecialCharTok{+}
  \FunctionTok{annotate}\NormalTok{(}\StringTok{"text"}\NormalTok{, }\AttributeTok{x =}\NormalTok{ .}\DecValTok{35}\NormalTok{, }\AttributeTok{y =}\NormalTok{ .}\DecValTok{45}\NormalTok{, }\AttributeTok{label =} \StringTok{"Good Three Point Shooters"}\NormalTok{, }\AttributeTok{col =}\StringTok{"red"}\NormalTok{, }\AttributeTok{size =} \DecValTok{4}\NormalTok{)}\SpecialCharTok{+}
  \FunctionTok{annotate}\NormalTok{(}\StringTok{"text"}\NormalTok{, }\AttributeTok{x =}\NormalTok{ .}\DecValTok{35}\NormalTok{, }\AttributeTok{y =}\NormalTok{ .}\DecValTok{1}\NormalTok{, }\AttributeTok{label =} \StringTok{"Average Shooter"}\NormalTok{, }\AttributeTok{col  =} \StringTok{"red"}\NormalTok{, }\AttributeTok{size =} \DecValTok{4}\NormalTok{)}\SpecialCharTok{+}
  \FunctionTok{annotate}\NormalTok{(}\StringTok{"text"}\NormalTok{, }\AttributeTok{x =}\NormalTok{.}\DecValTok{55}\NormalTok{, }\AttributeTok{y=}\NormalTok{.}\DecValTok{1}\NormalTok{, }\AttributeTok{label =} \StringTok{"Good Mid{-}range Shooter"}\NormalTok{, }\AttributeTok{col =} \StringTok{"red"}\NormalTok{, }\AttributeTok{size  =} \DecValTok{4}\NormalTok{)}
\end{Highlighting}
\end{Shaded}

\includegraphics{Report_files/figure-latex/unnamed-chunk-5-1.pdf}

The graph above shows shows a players three point percentage in relation
to their mid range percentage. The data points are random in nature with
each color representing the position that that player plays. The plot is
also broken up into four different quadrants to help viewers understand
how well a player shoots based on their position on the plot.

The initial thoughts from the graph is there will be little to no
correlation between a players ability to make a mid range shots and
their ability to make three pointers.

\begin{Shaded}
\begin{Highlighting}[]
\FunctionTok{cor}\NormalTok{(filtered\_mydata4}\SpecialCharTok{$}\NormalTok{ten\_sixteenfgperbyDistance, filtered\_mydata4}\SpecialCharTok{$}\NormalTok{threep\_fgperbyDistance)}
\end{Highlighting}
\end{Shaded}

\begin{verbatim}
## [1] 0.1603543
\end{verbatim}

The correlation coefficient is .160, way to low to infer that there is
any association between the two variables.

\subsubsection{\texorpdfstring{\textbf{Question \#3: Does a player's
position have any influence on their ability to hit three
pointers?}}{Question \#3: Does a player's position have any influence on their ability to hit three pointers?}}\label{question-3-does-a-players-position-have-any-influence-on-their-ability-to-hit-three-pointers}

Finally, we will look at a player's position on the court, and if it has
any influence on their ability to shoot three pointers. Does a player's
position influence their training enough where one position would shoot
much better than an other?

\begin{Shaded}
\begin{Highlighting}[]
\FunctionTok{ggplot}\NormalTok{(filtered\_mydata1, }\FunctionTok{aes}\NormalTok{(}\AttributeTok{x =} \FunctionTok{reorder}\NormalTok{(Pos, }\SpecialCharTok{{-}}\NormalTok{threeP\_percent), }\AttributeTok{y =}\NormalTok{ threeP\_percent))}\SpecialCharTok{+}
  \FunctionTok{geom\_boxplot}\NormalTok{()}\SpecialCharTok{+}
  \FunctionTok{ggtitle}\NormalTok{(}\StringTok{"Three Pointers v. Position"}\NormalTok{)}\SpecialCharTok{+}
  \FunctionTok{xlab}\NormalTok{(}\StringTok{"Three Point Percentage"}\NormalTok{)}\SpecialCharTok{+}
  \FunctionTok{ylab}\NormalTok{(}\StringTok{"Position"}\NormalTok{)}\SpecialCharTok{+}
  \FunctionTok{coord\_flip}\NormalTok{()}
\end{Highlighting}
\end{Shaded}

\includegraphics{Report_files/figure-latex/unnamed-chunk-7-1.pdf}

The graph above uses the same filtered data from question \#1. These box
plots help demonstrate the spread of data, but ultimately this allows us
to compare averages across each position.

Below is a table of the average three point percentage by position.

\begin{longtable}[]{@{}ll@{}}
\toprule\noalign{}
Position & Average Three point percentage \\
\midrule\noalign{}
\endhead
\bottomrule\noalign{}
\endlastfoot
PG & .368 \\
PF & .368 \\
SG & .365 \\
SF & .356 \\
C & .335 \\
\end{longtable}

The dark lines within each box represents the average three point
percentage from each position. The box represents the Interquartile
range, or the 50\% of the data. The the dark lines that look like tails,
are the upper and lower limits of the data. To no surprise, the best
shooting position is point-guard(PG), and Center(C) in last place. The
second best three point shooting position is PF, but they are a position
with the most spread.

This plot does not however demonstrate whether there is an association
between position and three point shooting. To do that we will use the
Kruskal-Wallis test. This test allows to determine if there are
statically significant differences between two or more groups. In our
case the groups are positions on a basketball court, and the dependent
variables are the shooting percentages of those positions.

\begin{Shaded}
\begin{Highlighting}[]
\FunctionTok{kruskal.test}\NormalTok{(filtered\_mydata1}\SpecialCharTok{$}\NormalTok{threeP\_percent}\SpecialCharTok{\textasciitilde{}}\NormalTok{filtered\_mydata1}\SpecialCharTok{$}\NormalTok{Pos)}
\end{Highlighting}
\end{Shaded}

\begin{verbatim}
## 
##  Kruskal-Wallis rank sum test
## 
## data:  filtered_mydata1$threeP_percent by filtered_mydata1$Pos
## Kruskal-Wallis chi-squared = 6.5341, df = 4, p-value = 0.1627
\end{verbatim}

From this test, we are returned several values but the one that is most
important is the one titled ``p-value''. The p-value is a number that
describes how significantly likely a participle set of observations are
to occur given the null hypothesis to be true. What we are looking for
here is for our p-value to be less than .05. To keep things simple, the
p-value for this experiment needs to be less than .05 for us to
confidently say that the position of a player significantly impacts
their ability to make three pointers. The p-value for our test is .1627,
greater than .05, so no. There is no significant impact on three point
shooting based on the position of a player

\subsubsection{\texorpdfstring{\textbf{Rebuttal}}{Rebuttal}}\label{rebuttal}

There are several limitations to this study that need to be addressed.
First, there is a lack of data. There are limitations to using only one
season. There could be contributing factors that influence the data. For
example, the Covid lock down season or even shortened seasons. Secondly,
correlation coefficients has several pitfalls and does not perfectly
describe the relationship between two variables. As the famous saying
goes, ``correlation does not imply causation''.

\subsubsection{\texorpdfstring{\textbf{Conclusion}}{Conclusion}}\label{conclusion}

In conclusion, this report aimed to determine if any statistical data
could evaluate a players ability to make three pointers at a high level.
This report looked at common associations between variables such as
free-trow and mid-range shooting, and uncommon associations like player
positions.

Making three pointers at a high level is a difficult task. There is a
lot that goes into it, and ultimately no data point can determine how
well someone can make three pointers better than the eye test. To truly
understand a player, you must watch and appreciate what they do.
Basketball is a beautiful sport, and it is best enjoyed when one can be
part of it.

\end{document}
